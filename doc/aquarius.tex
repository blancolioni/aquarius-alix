\documentclass{article}
\usepackage{verbatim}
\usepackage{makeidx}
\usepackage{listings}
\title{The Aquarius Book}
\author{Fraser Wilson}
\makeindex

\begin{document}
\maketitle
\tableofcontents
\listoffigures
\listoftables
\lstset{language=Ada}

\begin{abstract}
Aquarius is a source code processing tool.  Given an
ambiguous\footnote{with certain restrictions} EBNF description of a
language, Aquarius can parse it.  Adding layout rules allows Aquarius
to format it, for pretty-printing or editing.  Code in the target
language can be easily generated programmatically, or existing code
can be updated without losing changes.  Adding semantic rules
lets Aquarius check it for errors.  Finally, adding some abstract
coding rules allows Aquarius to generate code for a wide variety of
architectures, ranging from the 6502/6510, through the PDP-11, i386
and x86\_64.
\end{abstract}

\section{Introduction}
Aquarius is a source code processing tool.  Given an
ambiguous\footnote{with certain restrictions} EBNF description of a
language, Aquarius can parse it.  Adding layout rules allows Aquarius
to format it, for pretty-printing or editing.  Adding semantic rules
lets Aquarius check it for errors.  Finally, adding some abstract
coding rules allows Aquarius to generate code for a wide variety of
architectures, ranging from the 6502/6510, through the PDP-11, i386
and x86\_64.




\section{Conclusion}

Yep.

\bibliographystyle{alpha}
\bibliography{rose}

\printindex

\end{document}
